\documentclass[12pt,a4paper,parskip=full]{scrartcl}

\usepackage[utf8]{inputenc} 
\usepackage[ngerman]{babel} 
\usepackage{amsmath} 
\usepackage{graphicx} 
\usepackage{hyperref}
\usepackage{multicol}

\usepackage{biblatex}
\addbibresource{Proposal.bib}

\title{Entwicklung einer KI-gesteuerten Erweiterung zur Umwandlung von Audioaufnahmen in verarbeitbare Objekte für eine Verwaltungssoftware}
\author{Nathan Gebremichael}
\newcommand{\firma}{mediendesign AG} 
\date{}

\begin{document}

\makeatletter

{
    \centering
    {\Large Proposal zur Bachelorarbeit \par}
    \vspace*{1cm}
    {\LARGE \@title \par}
    \vspace*{1cm}
    {\large Firma: \firma \par}
    \vspace*{0.5cm}
    {\textit{\small von \@author} \par}
    }

\section*{Betreuung}
\begin{itemize}
    \item Prof. Dr. Oliver Hofmann, Betreuender Dozent der Technischen Hochschule Nürnberg
    \item Christian Koch, Informatiker, Lead Senior Developer bei mediendesign AG
\end{itemize}

% \begin{multicols}{2}
\section*{Ausgangssituation}
Die Ausgangssituation dieses Projekts ist eine von mediendesign AG-entwickelten Verwaltungssoftware, die von verschiedenen Versicherungen zur Verwaltung von Schadensfällen verwendet wird.
Sacharbeiter aus diesen Versicherungen können dabei neue Schadensmeldungen aufnehmen und bearbeiten, die relevanten Informationen in Formulare eintragen und im System abspeichern.

Eine mögliche Erleichterung dieses Prozesses könnte die Verwendung von Sprachaufnahmen sein, die der Bearbeiter vor Ort bei der Besichtigung des Schadensobjekts erstellt.
Die Idee ist, diese Audioaufnahmen durch Künstliche Intelligenz (KI) in ein Objekt umzuwandeln, das dann im Backend der Verwaltungssoftware weiterverarbeitet werden kann.
Dies würde den Prozess der Dateneingabe erheblich vereinfachen und gleichzeitig beschleunigen, da der Bearbeiter vor Ort nur eine Aufnahme erstellen und abschicken muss und der Eintrag automatisch in das System abgelegt wird.

Die Herausforderung besteht darin, eine Lösung zu entwickeln, die in der Lage ist, die Audioaufnahmen effizient und genau zu transkribieren und in ein geeignetes Format für die Backend-Verarbeitung umzuwandeln.
Darüber hinaus muss diese Lösung lokal gehostet werden können und lokale Modelle verwenden, um Datenschutz- und Sicherheitsanforderungen zu erfüllen.

\section*{Zielsetzung}
Das Hauptziel dieses Projekts ist die Entwicklung einer KI-gesteuerten Erweiterung für eine Verwaltungssoftware, die in der Lage ist, Audioaufnahmen eines Bearbeiters entgegenzunehmen und diese automatisch in ein Format umzuwandeln, das von der Software verarbeitet werden kann.
Dies soll erreicht werden, indem die Audioaufnahmen in Text transkribiert und dann in verarbeitbare Objekte umgewandelt wird.

% Ein weiteres Ziel ist die Verwendung von lokalen Modellen zur Durchführung der Transkription und Umwandlung, um die Anforderung des lokalen Hostings zu erfüllen. Dies stellt sicher, dass die Anwendung auch in Umgebungen funktioniert, in denen eine Internetverbindung nicht immer garantiert ist.

% Darüber hinaus zielt das Projekt darauf ab, die Effizienz und Produktivität der Verwaltungssoftware zu verbessern. Durch die Automatisierung des Prozesses der Audioaufnahme und -verarbeitung wird die manuelle Eingabe reduziert, was zu einer Zeitersparnis für die Benutzer führt.

% Schließlich ist es ein Ziel dieses Projekts, eine robuste und zuverlässige Anwendung zu entwickeln, die unter verschiedenen Bedingungen gut funktioniert. Dies beinhaltet umfangreiche Tests und Optimierungen, um sicherzustellen, dass die Anwendung den Anforderungen der Benutzer gerecht wird.

\section*{Vorgehensweise}
Die Entwicklung einer KI-gesteuerten Erweiterung zur Umwandlung von Audioaufnahmen in verarbeitbare Objekte für Verwaltungssoftware erfordert eine sorgfältige Planung und Durchführung.
Der erste Schritt in diesem Prozess ist die Anforderungsanalyse und Planung.
Hierbei werden die genauen Anforderungen und Ziele des Projekts definiert, wobei besonderes Augenmerk auf die Beschränkungen und Anforderungen, wie die lokale Hosting-Anforderung und die Verwendung lokaler Modelle, gelegt wird.

Nachdem die Anforderungen klar definiert sind, folgt die Auswahl des Modells.
Die Huggingface-Modelbibliothek bietet eine Vielzahl von Modellen, die speziell für Sprachverarbeitungsaufgaben wie Transkription oder Textgenerierung entwickelt wurden. \cite{huggingface_hugging_nodate}
Ein passendes Modell wird ausgewählt und für die Integration in die Anwendung vorbereitet.

Der nächste Schritt ist die Entwicklung der Anwendung selbst. 
Dies umfasst die Einrichtung eines Servers für das Hosting der Anwendung, die Implementierung der Audioaufnahme-Funktionalität und die Integration des ausgewählten Modells. 
Besondere Aufmerksamkeit wird auf die Integration von LangChain \cite{langchain_agents_nodate} oder AutoGen \cite{microsoft_getting_nodate} gelegt, um mit Agenten und Promt Engineering die Audioaufnahmen in Text umzuwandeln.

Sobald die Anwendung entwickelt ist, wird sie umfangreichen Tests unterzogen.
Diese Tests stellen sicher, dass alle Komponenten korrekt interagieren und dass die Anwendung wie erwartet funktioniert. Eventuell notwendige Optimierungen werden in diesem Schritt durchgeführt.

Schließlich, nachdem sichergestellt wurde, dass alles korrekt funktioniert, wird die Anwendung bereitgestellt.
Dieser Schritt markiert den Abschluss des Projekts und den Beginn der Nutzung der Anwendung durch Endbenutzer.

% \end{multicols}
\printbibliography


\end{document}